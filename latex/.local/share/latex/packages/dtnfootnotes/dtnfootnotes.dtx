% \iffalse meta-comment
%
% Copyright (C) 2023 by David Nebauer <david@nebauer.org>
% ---------------------------------------------------------------------------
% This work may be distributed and/or modified under the
% conditions of the LaTeX Project Public License, either version 1.3
% of this license or (at your option) any later version.
% The latest version of this license is in
%   http://www.latex-project.org/lppl.txt
% and version 1.3 or later is part of all distributions of LaTeX
% version 2005-12-01 or later.
%
% This work has the LPPL maintenance status `maintained'.
%
% The Current Maintainer of this work is David Nebauer.
%
% This work consists of the files dtnfootnotes.dtx and dtnfootnotes.ins
% and the derived filebase dtnfootnotes.sty.
%
% \fi
%
% \iffalse
%<*driver>
\ProvidesFile{dtnfootnotes.dtx}
%</driver>
%<package>\NeedsTeXFormat{LaTeX2e}[2022-11-01]
%<package>\ProvidesPackage{dtnfootnotes}
%<*package>
    [2023-07-10 v0.1 Custom footnote formatting]
%</package>
%
%<*driver>
\documentclass{ltxdoc}
\usepackage{dtnfootnotes}[2023-07-10]
\usepackage{xltabular,bigstrut}
\usepackage{enumitem}
\usepackage{parskip}
\frenchspacing
\newcommand{\pkgname}[1]{\textsf{#1}}
\newcommand{\optname}[1]{\texttt{#1}}
\newcommand{\optval}[1]{\texttt{#1}}
\newcommand{\codesep}{\vspace{2ex}}
\BeforeBeginEnvironment{macrocode}{\codesep}
\AfterEndEnvironment{macrocode}{\codesep}
\EnableCrossrefs
\CodelineIndex
\RecordChanges
\begin{document}
  \DocInput{dtnfootnotes.dtx}
  \PrintChanges
  \PrintIndex
\end{document}
%</driver>
% \fi
%
% \changes{v0.1}{2023-07-10}{Converted to DTX file}
%
% \DoNotIndex{\newcommand,\newenvironment}
%
% \providecommand*{\url}{\texttt}
% \GetFileInfo{dtnfootnotes.dtx}
% \title{The \textsf{dtnfootnotes} package}
% \author{David Nebauer \\ \url{david@nebauer.org}}
% \date{\fileversion~from \filedate}
%
% \maketitle
%
% \section*{Introduction}
%
% The default formatting of footnotes in \LaTeX{} is painfully ugly with
% footnotes indented and footnote marks superscripted and directly adjoining
% the footnote text.
%
% This package changes the footnote format to:
%
% \begin{itemize}[nosep,leftmargin=*]
% \item flush left
% \item display the mark in normal (non-superscripted) font
% \item add space between the footnote mark and text.
% \end{itemize}
%
% The package can also optionally:
%
% \begin{itemize}[nosep,leftmargin=*]
% \item add a blank line between footnotes
% \item adjust the space between the footnote mark and text
% \item adjust the footnote font size.
% \end{itemize}
%
% \section*{Usage}
%
% The package is loaded with the command:
% \begin{quote}
%    \cmd{\usepackage}\oarg{sep}\oarg{margin}\oarg{size}\verb|{dtnfootnotes}|
% \end{quote}
% The package options are:
%
% \setlength{\bigstrutjot}{6pt}
% \newcommand{\fsize}[1]{\texttt{\textbackslash{}#1}}
% \begin{xltabular}{\linewidth}{rX}
%        \optname{sep} & This is a boolean that determines whether footnotes
%                        are separated by a blank line. It defaults to
%                        \optval{false}. The option value is stored in the
%                        \TeX{} switch \texttt{@dtn@opt@sep}.
%                        \bigstrut[b] \\
%     \optname{margin} & This accepts a length used to set
%                        \fsize{footnotemargin}, which determines the space
%                        between the footnote mark and text. It defaults to
%                        \optval{1.10em}. If an invalid length is provided
%                        the package displays a warning and reverts
%                        \optname{margin} to its default value. The option
%                        value is stored in the \fsize{dtn@opt@margin} macro.
%                        \bigstrut[b] \\
%       \optname{size} & This is a font size used to set \fsize{footnotesize},
%                        which is one of the standard \LaTeX{} predefined
%                        commands: \fsize{tiny}, \fsize{scriptsize},
%                        \fsize{footnotesize}, \fsize{small},
%                        \fsize{normalsize}, \fsize{large}, \fsize{Large},
%                        \fsize{LARGE}, \fsize{huge}, and \fsize{Huge}. Use
%                        the option value \optval{\scriptsize} to reduce the
%                        footnote size by one step, and \optval{\small} to
%                        increase it by one step. The option value is stored
%                        in the \fsize{dtn@opt@size} macro.
% \end{xltabular}
%
% Here is an example in which all options are used in loading the package:
% \begin{quote}
%    \cmd{\usepackage}\verb|[sep=true][margin=1.6em][size=\small]{dtnfootnotes}|
% \end{quote}
%
% \section*{Implementation}
%
% \iffalse
%<*package>
% \fi
%
%    \begin{macrocode}
\DeclareCurrentRelease{v0.1}{2023-07-10}
\NeedsTeXFormat{LaTeX2e}
\ProvidesPackage{dtnfootnotes}
    [2023-07-10 v0.1 Improve footnote formatting]
%    \end{macrocode}
%    \DescribeMacro{@dtn@opt@sep} \DescribeMacro{\dtn@opt@margin}
%    \DescribeMacro{\dtn@opt@size} Process the package options.
%    \begin{macrocode}
\DeclareKeys{
          sep.if = @dtn@opt@sep,        sep.usage = preamble,
    margin.store = \dtn@opt@margin,  margin.usage = preamble,
      size.store = \dtn@opt@size,      size.usage = preamble
}
\SetKeys{
       sep = false,
    margin = 1.10em,
      size = \empty
}
\ProcessKeyOptions
%    \end{macrocode}
%    Processing the \optname{margin} option value requires checking the
%    validity of a length value. This is done by setting up a command that
%    checks length validity. The command definition is taken from
%    \url{https://tex.stackexchange.com/a/499223}. It, in turn, credits
%    \url{https://stackoverflow.com/a/23872060/923955} for the regex it
%    contains.
%
%    To ensure that any commands containing lengths, such as \cmd{\small}, are
%    expanded the newly defined command needs to be invoked with
%    \cmd{\expandafter} preceding and following the command name.
%    \begin{macrocode}
\ExplSyntaxOn
\NewDocumentCommand{\iflengthTF}{mmm}{%
    \regex_match:nnTF
    { \A [+\-]? ((\d+(\.\d*)?)|(\.\d+)) \s*
    (pt|pc|in|bp|cm|mm|dd|cc|sp|ex|em) \Z} % regex
    { #1 } % test string
    { #2 } % true text
    { #3 } % false text
}
\ExplSyntaxOff
%    \end{macrocode}
%    If the \optname{margin} is not a valid length then warn the user and reset
%    \optname{margin} to the default value.
%    \begin{macrocode}
\expandafter\iflengthTF\expandafter{\dtn@opt@margin}{}{%
    \PackageWarning{dtnfootnotes}{margin value is not a valid length
                                  \MessageBreak
                                  using default value of 1.10em instead}
    \renewcommand{\dtn@opt@margin}{1.10em}
}
%    \end{macrocode}
%    Uses \pkgname{footmisc} to shift the footnotes flush left and use hanging
%    indent. The \pkgname{ifthen} package is used for logical branching. The
%    purpose served by the \pkgname{setspace} is truly mysterious.
%    \begin{macrocode}
\RequirePackage[hang,flushmargin,splitrule,multiple,abovefloats]{footmisc}
\RequirePackage{etoolbox}
\RequirePackage{ifthen}
\RequirePackage{setspace}
%    \end{macrocode}
%    Change the footnote font size if a \optname{size} option value was
%    provided.
%    \begin{macrocode}
\ifthenelse{\equal{\dtn@opt@size}{\empty}}{}{%
    \renewcommand{\footnotesize}{\small}
}%
%    \end{macrocode}
%    Make the footnote mark normal font size. This technique is taken from
%    \url{https://tex.stackexchange.com/a/306278}.
%    \begin{macrocode}
\patchcmd{\@makefntext}{%
\ifFN@hangfoot
\bgroup}%
{%
\ifFN@hangfoot
\bgroup\def\@makefnmark{\rlap{\normalfont\@thefnmark.}}}{}{}%
\patchcmd{\@makefntext}{%
\ifdim\footnotemargin>\z@
\hb@xt@ \footnotemargin{\hss\@makefnmark}}%
{%
\ifdim\footnotemargin>\z@
\hb@xt@ \footnotemargin{\@makefnmark\hss}}{}{}%
%    \end{macrocode}
%    Set distance between main text and note rule.
%    \begin{macrocode}
\setlength{\skip\footins}{1\baselineskip}
%    \end{macrocode}
%    Set distance between footnote mark and text, using the \optname{margin}
%    option value if provided.
%    \begin{macrocode}
\expandafter\setlength\expandafter{\footnotemargin}{\dtn@opt@margin}
%    \end{macrocode}
%    Add a blank line between footnotes if requested with the \optname{sep}
%    option. This is implemented by setting \cmd{\footnotesep} to a non-zero
%    length. This vertical space is inserted before each footnote.
%    \begin{macrocode}
\ifthenelse{\boolean{@dtn@opt@sep}}{%
    \setlength{\footnotesep}{\skip\footins} % Between footnotes [= previous]
}{}%
\renewcommand{\hangfootparskip}{0pt}
\renewcommand{\hangfootparindent}{1em}
%    \end{macrocode}
%
% \iffalse
%</package>
% \fi
\endinput
